\documentclass[twoside,twocolumn,10pt]{extarticle}

\usepackage{multirow}
\usepackage{textcase}
\usepackage{amsthm}
\usepackage{amssymb}
%\newtheorem{thm}{Theorem}
\theoremstyle{definition}
%\newtheorem{defn}[thm]{Definition} % definition numbers are dependent on theorem numbers
%\newtheorem{exmp}[thm]{Example}

\usepackage{blindtext} % Package to generate dummy text throughout this template 

\usepackage[sc]{mathpazo} % Use the Palatino font
\usepackage[T1]{fontenc} % Use 8-bit encoding that has 256 glyphs
\linespread{1.05} % Line spacing - Palatino needs more space between lines
\usepackage{microtype} % Slightly tweak font spacing for aesthetics

\usepackage[italian]{babel} % Language hyphenation and typographical rules
\usepackage[utf8]{inputenc}

\usepackage[hmarginratio=1:1,top=32mm,columnsep=20pt]{geometry} % Document margins
\usepackage[hang,small,labelfont=bf,up,textfont=it,up]{caption} % Custom captions under/above floats in tables or figures
\usepackage{booktabs} % Horizontal rules in tables
\usepackage{subcaption}

\usepackage{graphicx}
\usepackage{float}
\usepackage{listings}
\usepackage{color}
\usepackage{textcomp}

\definecolor{codegreen}{rgb}{0,0.6,0}
\definecolor{codegray}{rgb}{0.5,0.5,0.5}
\definecolor{codepurple}{rgb}{0.58,0,0.82}
\definecolor{backcolour}{rgb}{0.95,0.95,0.92}

\lstdefinestyle{mystyle}{
	backgroundcolor=\color{backcolour},   
	commentstyle=\color{codegreen},
	keywordstyle=\color{magenta},
	numberstyle=\tiny\color{codegray},
	stringstyle=\color{codepurple},
	basicstyle=\footnotesize,
	breakatwhitespace=false,         
	breaklines=true,                 
	captionpos=b,                    
	keepspaces=true,                 
	numbers=left,                    
	numbersep=5pt,                  
	showspaces=false,                
	showstringspaces=false,
	showtabs=false,                  
	tabsize=2
}
\lstset{style=mystyle}

\usepackage{lettrine} % The lettrine is the first enlarged letter at the beginning of the text

\usepackage{enumitem} % Customized lists
\setlist[itemize]{noitemsep} % Make itemize lists more compact

\usepackage{abstract} % Allows abstract customization
\renewcommand{\abstractnamefont}{\normalfont\bfseries} % Set the "Abstract" text to bold
\renewcommand{\abstracttextfont}{\normalfont\small\itshape} % Set the abstract itself to small italic text

\usepackage{titlesec} % Allows customization of titles
\renewcommand\thesection{\Roman{section}} % Roman numerals for the sections
\renewcommand\thesubsection{\roman{subsection}} % roman numerals for subsections
\titleformat{\section}[block]{\large\scshape\centering}{\thesection.}{1em}{} % Change the look of the section titles
\titleformat{\subsection}[block]{\large}{\thesubsection.}{1em}{} % Change the look of the section titles

\usepackage{fancyhdr} % Headers and footers
\pagestyle{fancy} % All pages have headers and footers
\fancyhead{} % Blank out the default header
\fancyfoot{} % Blank out the default footer
\fancyhead[C]{Apprendimento per Rinforzo $\bullet$ Maggio 2017 $\bullet$ \textit{Machine Leargning}} % Custom header text
\fancyfoot[RO,LE]{\thepage} % Custom footer text

\usepackage{titling} % Customizing the title section

\usepackage{hyperref} % For hyperlinks in the PDF

% Title Section
\setlength{\droptitle}{-4\baselineskip} % Move the title up

\pretitle{\begin{center}\Huge\bfseries} % Article title formatting
\posttitle{\end{center}} % Article title closing formatting
\title{Reinforcement Learning e\\ Agenti basati su Experience Replay:\\ Caso di Studio di Inefficacia} % Article title
\author{%
\textsc{Maxim Gaina e Bartolomeo Lombardi} \\[1ex] % Your name
\normalsize Università di Bologna\thanks{Progetto per il corso di Complementi di Linguaggi di Programmazione, A.A. 2016/2017, prof. Andrea Asperti.} \\ % Your institution
\normalsize \href{mailto:maxim.gaina@studio.unibo.it}{\{maxim.gaina, bartolomeo.lombardi\}@studio.unibo.it}
%\and % Uncomment if 2 authors are required, duplicate these 4 lines if more
%\textsc{Jane Smith}\thanks{Corresponding author} \\[1ex] % Second author's name
%\normalsize University of Utah \\ % Second author's institution
%\normalsize \href{mailto:jane@smith.com}{jane@smith.com} % Second author's email address
}
\date{\today} % Leave empty to omit a date
\renewcommand{\maketitlehookd}{%
\begin{abstract}
\noindent L'obiettivo di questo lavoro di progetto consiste in primo luogo nello studio di una particolare tecnica di apprendimento, detta \textit{Reinforcement Learning} applicata all'ambito ludico.
\end{abstract}
}

\begin{document}

\maketitle

\tableofcontents

\section*{Introduzione}
	\lettrine[nindent = 0.4em,lines=3]{O}\space\MakeTextLowercase{g}ni lingua 
	
\section{Agente Deep Q\texttwelveudash Network}\label{sec:dqn-agent}
	\theoremstyle{plain}
	\newtheorem{definition}{Definizione}
	\begin{definition}[Def]\label{def:}
		def
	\end{definition}

\section{Sotto la Soglia Umana}\label{}
	
\section{Le prove}\label{}
	\subsection{Le features}
		\begin{table*}[h]
			\centering
			\caption{Gerarchia dei fenomeni coinvolti nel riconoscimento automatico della Prominenza}
			
			\label{tab:gerarchia}
		\end{table*}
		\begin{table*}[h]
			\centering
			\caption{Il migliore sistema \texttt{PGM} (\textit{Latent-Dynamic Conditional Neural Fields}) al variare del metodo di selezione delle sillabe prominenti.}
			\label{tab:pgm}
			
		\end{table*}
	
\section{Agenti Alternativi}\label{}
	\begin{figure*}[h]
		\centering
		%\includegraphics[scale=.4]{img/rnn.png}
		\caption{\textit{Unrolling} di una rete neurale ricorsiva}
		\label{fig:unroll}
	\end{figure*}

	\begin{figure*}[h]
		\centering
		%\includegraphics[scale=.5]{img/lstm.png}
		\caption{Esempio di struttura interna alla base delle \textit{Long Short Term Memory (\texttt{LSTM})}}
		\label{fig:lstm}
	\end{figure*}

	\begin{figure*}[h]
		\centering
		\begin{subfigure}[b]{.496\textwidth}
			%\includegraphics[width=\textwidth]{img/inside-lstm.png}
			\caption{Parametri calcolati nelle prime fasi}
		\end{subfigure}
		\begin{subfigure}[b]{.496\textwidth}
			%\includegraphics[width=\textwidth]{img/inside-lstm2.png}
			\caption{Parametri calcolati nelle ultime fasi}
		\end{subfigure}
		\caption{Mappa dei parametri interni a una \texttt{LSTM}}
		\label{fig:parmap}
	\end{figure*}
	
	\begin{equation}\label{eq:ft}
		f_t = \sigma(W_f \cdot [h_{t - 1}, x_t] + b_f)
	\end{equation}
	
	\begin{figure*}[h]
		\centering
		%\includegraphics[scale=.3]{img/seqs.jpeg}
		\caption{Problemi \textit{sequence-to-sequence} trattabili con le \texttt{RNN}}
		\label{fig:seqs}
	\end{figure*}
	\begin{figure}
		\centering
		%\includegraphics[scale=.26]{img/blstm.png}
		\caption{Architettura astratta di una rete \texttt{BLSTM}}
		\label{fig:blstm}
	\end{figure}
	\begin{figure*}[ht!]
		\centering
		\caption{Costruzione del modello con \texttt{Keras}}
		%\lstinputlisting[language=Python]{code/seq.py}
		\label{fig:modelcode}
	\end{figure*}
	\begin{table*}[]
		\centering
		\caption{Risultati ottenuti in seguito al lavoro di progetto, confrontati con il modello \texttt{LDCNF}.}
		\label{tab:results}
		
	\end{table*}
	
	\section{Conclusioni}	
	
\begin{thebibliography}{99}	
	\bibitem{bib:fenomeni-prosodici-prominenza}
		Fabio Tamburini,
		\newblock \emph{Fenomeni Prosodici e Prominenza: Un Approccio Acustico},
		2005.

	\bibitem{bib:prominence-detection-italian}
		Fabio Tamburini, Chiara Bertini, Pier Marco Bertinetto,
		\newblock \emph{Prosodic prominence detection in Italian continuous speech using probabilistic graphical models},
		2014

	\bibitem{bib:prominence-by-acoustic-analyses}
		Fabio Tamburini,
		\newblock \emph{Automatic Detection of Prosodic Prominence by Means of Acoustic Analyses},
		2015

	\bibitem{bib:chollet2015keras}
		Chollet, Fran\c{c}ois,
		\newblock \emph{Keras},
		\url{https://github.com/fchollet/keras},
		2015

	\bibitem{bib:blstm}
		Alex Graves,
		\newblock \emph{Supervised Sequence Labelling with Recurrent Neural Networks},
		2008
\end{thebibliography}

\end{document}
